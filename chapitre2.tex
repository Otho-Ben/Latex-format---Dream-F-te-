%% @Author: Ines Abdeljaoued Tej
%  @Date:   2018-06
%% @Class:  Graduation Project, ESSAI - Carthage University, Tunisia.
  \setcounter{section}{0} % Reset section counter to 0
\renewcommand{\thesection}{\arabic{section}} % Redefine section numbering to arabic numerals
\setcounter{subsection}{0} % Reset subsection counter to 0
L'Entreprise Safari vient tout juste de commencer son activité dans le monde de l'industrie et de l'informatique. Ainsi voici son cahier des charges :
\section{Utilisateur :}

L'utilisateur a le plein droit de créer, supprimer, se connecter, se déconnecter de son compte. Chaque utilisateur dispose d'un identifiant unique qui permet au système de l'identifier lors des différentes modifications, il est caractérisé par son nom, prénom, email, mot de passe et d'un droit ( utilisateur normal - super utilisateur - admin utilisateur). \\

L’utilisateur doit s’authentifier avant toute actions pour maximiser la sécurité 
de l’application, il s’identifie grâce à un nom \& mot de passe. 

\section{Administrateur :}

L'administrateur est la personne qui prend en charge toute ce qui sa passe au sein de "Safari", et assure que toutes règles soient respectés. L'administrateur a le droit de gérer les utilisateurs ( supprimer, créer, modifier), de même pour les postes. \\

L’Admin doit s’authentifier avant toute actions pour maximiser la sécurité 
de l’application, il s’identifie grâce à un nom \& mot de passe.  
L'administrateur est caractérisé par son id, nom, prénom et email.

\section{Blog ( Le poste ) :}

Chaque poste doît seulement avoir des informations essentiels comme le titre, le contenue , la date de création ainsi que l'image du profile de l'utilisateur.

